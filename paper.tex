\documentclass[12pt,letter]{article}
%\pagestyle{headings}
%\usepackage[francais]{babel}
\usepackage[latin1]{inputenc}		% comme fontenc T1 mais avec meilleur rendu visuel
\usepackage{amsmath}
\usepackage{amsfonts}
\usepackage{amssymb}
\usepackage{amsthm}


\usepackage{graphicx}			% pour ins�rer des images
%\usepackage{color}			% pour g�rer les sorties en couleur (xfig notamment)
\usepackage{indentfirst}		% pour indenter le premier paragraphe apr�s un titre
%\usepackage{thloria2/tlfloat}
%\usepackage{float}			% pour placer les figures o? on veut
%\usepackage{floatflt}			% pour placer les figures o? on veut
%\usepackage{textcomp}
%\usepackage{makeidx}
%\usepackage[Lenny]{/sw/share/texmf-dist/tex/latex/fncychap/fncychap}
%\usepackage{algorithmic}
%\usepackage{algorithm}
\usepackage{hyperref}
\usepackage{verbatim}

\bibliographystyle{siam}

\usepackage{layout}

%Modification structure
\setlength{\oddsidemargin}{0cm}
\setlength{\marginparwidth}{0cm}
\addtolength{\textwidth}{60pt}
\addtolength{\textheight}{1cm}

%\renewcommand{\baselinestretch}{2}

\newcommand{\powset}[1]{\ensuremath{\mathcal{P}(#1)}}
\newcommand{\tf}{ \ensuremath{\therefore} }

%\makeindex

\title{PetitML}
\author{Antoine \textsc{Hersen}}

\begin{document}

\maketitle
\newpage

\tableofcontents
\newpage

%\layout

\section{Introduction}
We have implemented a small functional language. The compiler emit \texttt{PPC} assembly code that is linked with a small runtime library in \texttt{C}.

\section{Problems encountered}
The most difficult part was to conform to the \texttt{System V PPC} application binary interface in order to be able to take advantage of \texttt{C} functions.

\section{Lesson learned}
The programme make use of monad in several occasion. The obvious case as alternate mean of computation in the intermediate interpreter present for testing, but also for AST traversal when carrying states.

\section{Programme}
The programme is divided in different modules.

\subsection{Syntax}
The syntax is described in this module.
\verbatiminput{Syntax.hs}

\subsection{K-normal}
Apply the first normalisation. Contrary to the common choice of the CPS transformation K-normal does not produce a large number of administrative lambdas.

\verbatiminput{Knormal.hs}

There is an interpreter for K-normalised programme for testing purpose.

\verbatiminput{Kinterp.hs}

\subsection{Eta}
The eta transformation facilitate the programme manipulation.

\verbatiminput{Eta.hs}

\subsection{Closure}
The closure conversion is necessary as function are first order.

\verbatiminput{Closure.hs}

\subsection{PPC asm}
Generation of PPC assembly code.

\verbatiminput{PPCasm.hs}

\subsection{PetitML}
Some function used across the different modules.

\verbatiminput{PetitML.hs}

\subsection{Runtime}
The runtime is minimal.

\verbatiminput{asm/simple.c}

\subsection{Testing}

\verbatiminput{Testing.hs}

\section{Example}

Output example for the programme 3 described in the Testing module.

\begin{verbatim}
  .file   "petitML.pml"
   .section        ".text"
   .align 2
   .global f1
   .type  , @function
f1:
   stwu 1,-120(1)
   mflr 0
   stw 0,124(1)
   mr 6,3
   mr 7,4
   lwz 8,4(3)
   mr 5,8
   mr 3,5
   lwz 2,0(1)
   lwz 0,4(2)
   mtlr 0
   mr 1,2
   blr
   .size  f1,.-f1
   .align 2
   .global petitML_entry
   .type  , @function
petitML_entry:
   stwu 1,-120(1)
   mflr 0
   stw 0,124(1)
   mr 31,3
   li  9,0
   mr 10,9
   cmpwi 3,10,0
   bne 3,.L1
   li  11,12
   li  12,1
   mr 15,31
   lis  2,f1@ha
   la 0,f1@l(2)
   stw 0,0(31)
   stw 11,4(31)
   addi 0,31,8
   mr 14,15
   li  16,2
   mr 15,16
   stmw 5,8(1)
   lwz 2,0(14)
   mtctr 2
   mr 4,15
   mr 3,14
   bctrl
   mr 0,31
   lmw 5,8(1)
   mr 31,0
   mr 13,3
   add 8,12,13
   b .L2
.L1:
   li  8,200
.L2:
   li  9,3
   mr 10,8
   add 5,9,10
   mr 3,5
   lwz 2,0(1)
   lwz 0,4(2)
   mtlr 0
   mr 1,2
   blr
   .size  petitML_entry,.-petitML_entry
   .ident  "Super petit Ml compiler"
   .section        .note.GNU-stack,"",@progbits

\end{verbatim}

\end{document}
